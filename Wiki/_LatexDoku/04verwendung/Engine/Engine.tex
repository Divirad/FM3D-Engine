\section{Engine}
\label{verwendung_engine}
Wenn Sie die FM3D-Engine in Kombination mit dem FM3D-Designer verwenden, so wird Ihnen bei der Projekterstellung ein funktionsfähiges VisualStudio C++ Projekt generiert, welches in der VisualStudio-Solution \textit{GameProject.sln} zu finden ist.
Es wird Ihnen geraten, nichts an dem generierten Code zu ändern.
In der Datei \textit{"`presets.h"'} werden die Entity-Presets generiert, welche Sie im Designer erstellen können.
Sie können dem Projekt auch neue Dateien hinzufügen und unabhängig vom generierten Code programmieren. Weitere Dateien, welche ursprünglich nicht zu dem generierten Projekt gehört haben, sollten keinen Einfluss auf die Funktionalität des generierten Codes haben. Das Spiel an sich steht in der \textit{Main.cpp} Datei, welche im Projekt bereits vorhanden ist.

\subsection{Voreinstellungen}
Falls Sie die Engine in Kombination mit dem FM3D-Designer verwenden, so werden die Verzeichnisse automatisch eingebunden. Falls nicht, so müssen Sie Zunächst die Bibliotheken der FM3D-Engine, OpenGL, FreeImage, FreeType und Assimp in das Projekt manuell einbinden. Fügen Sie die Verzeichnisse in die zugehörige Option hinzu. Das ganze sollte so in ihren Einstellungen aussehen:
$$Configuration Properties->C/C++->Additional Include Directories$$\cref{includeinc}
$$Configuration Properties->Linker->Additional Include Directories$$
\cref{liblib}

\begin{figure}
	\begin{center}
		\includegraphics[width=\textwidth]{04verwendung/Engine/include.png}
		\caption{Include Verzeichnisse}\label{includeinc}
	\end{center}
\end{figure}

\begin{figure}
	\begin{center}
		\includegraphics[width=\textwidth]{04verwendung/Engine/lib.png}
		\caption{Lib-Verzeichnisse}\label{liblib}
	\end{center}
\end{figure}

\subsection{Kamera}
Die Kamera ist in dem generierten Projekt bereits vorhanden. Ihnen wird empfohlen die Werte der Position, Rotation, Zoom usw. der Kamera im Bereich der \textit{Game-Logic} zu verändern. Auch dieser Bereich ist eindeutig mit Kommentaren kenntlich gemacht.
In dem erstellten Projekt existiert bereits ein Objekt der Klasse \textit{Camera}. Die \textit{Get}-Methoden geben eine Referenz zu dem jeweiligen Parameter der Kamera. Diese können somit sofort manipuliert werden. Als Beispiel:
\begin{lstlisting}{Cam}
// Create Camera Object
Camera camera(Vector3f(0.3f,0.4f,0.0f));
// Positionattribute Manipulation
camera.GetPosition().x += 0.5f;
camera.GetPosition().y -= 0.1f;
\end{lstlisting}
Für weitere Infos über die Syntax der Kamera wird empfohlen, in der DoxyGen-Dokumentation die Klasse "`Camera"' nachzuschlagen.

\subsection{Entities}
Erstellen Sie zunächst die Entities an der eindeutig kommentierten Stelle \textit{"'Create Entities here"`}. Erstellen Sie beliebig viele Preset-Objekte Ihrer erstellten Entity-Preset-Klassen. Sie können nun bei jedem beliebigen Objekt "`Preset-Einstellungen"' tätigen, in dem Sie dem Entity-Preset-Objekt Standardwerte zuweisen. Gehen wir davon aus, es gäbe ein Entitypreset Namens BaumPreset. Diesem wurde mittels Designer das RenderableComponent hinzugefügt.
\begin{lstlisting}{Entity}
BaumPreset baumpreset1;
\end{lstlisting}
Um nun dem Preset-Objekt ein Standartmodel zuzuweisen, verwendet man die folgende Syntax:

Erstellen wir zunächst ein Referenz-Objekt vom Typ Model.
\begin{lstlisting}{Entity}
Model* baummodel;
\end{lstlisting}
Um nun ein Model zu laden, wird die Klasse \textit{ExternFileManager} wie folgt verwendet:
\begin{lstlisting}{Entity}
ExternFileManager::
ReadModelFile("res/baum.dae", rendersystem,
&baummodel, false, true);
\end{lstlisting}
Dies sagt nun Folgendes aus:
wir übergeben den Dateipfad "`res/baum.dae"' und ein Objekt des Render-Systems, in dem dann das Model gerendert werden soll. Daraufhin wird eine Adresse zu dem Objekt der Klasse \textit{Model} übergeben, in der das Model gespeichert wird. Der darauffolgende boolesche Wert gibt an, ob Instancing verwendet werden soll und der folgende, ob das Modell animierbar sein soll. Sie können nun dem Objekt Baum dieses Standartmodel zuweisen. Dies geschieht wie folgt:
\begin{lstlisting}{Entity}
baumpreset1.SetModel(&baummodel);
\end{lstlisting}
Initialisieren Sie nun einen Entity-Pointer (Klasse: \textit{EntityPtr}) und erstellen Sie das Entity in der bereits initialisierten \textit{EntityCollection} namens \textit{scene}. Als Beispiel:
\begin{lstlisting}{Entity}
EntityCollection scene;
EntityPtr baum = scene.CreateEntity();
\end{lstlisting}
Ein Objekt der Klasse \textbf{EntityCollection} beinhaltet alle Entities, die in dem Spiel verwendet werden. Jedoch können auch mehrere EntityCollection erstellt und verwendet werden.
Um die Presets auf einen Entity-Pointer und somit auf ein Entity in einer EntityCollection anzuwenden, wird die folgende Syntax verwendet:
\begin{lstlisting}{Entity}
baumpreset1.SetComponents(baum);
\end{lstlisting}
Nun wird dem EntityPointer alle Komponenten, mit den zuvor zugewiesenen Standartwerten, zugewiesen.

Wenn ein Entity das Komponent \textit{RenderableComponent} besitzt, so kann es in dem eindeutig kommentierten Abschnitt "`Submit objects here to renderer"' dem Renderer übergeben werden. Das heißt nun, dass das Model des Entities gerendert werden kann.
\begin{lstlisting}{Entity}
/// ##########################
  //
  //  Submit objects here to renderer!
  //
  renderer3D-Submit(baum.get());
/// ##########################
\end{lstlisting}
Bevor Sie den Entities die Modelle zuweisen, vergessen Sie nicht die Modelle in die Projektmappe von VisualStudio zu laden.

\subsection{Inputsystem}
\label{inputsystemver}
Hierfür wird die Klasse \textit{Input} verwendet. Möchte man nun Tasten abfragen, so muss man zunächst über das Fenster auf das Objekt der Klasse Input zugreifen. 
Nun kann eine Methode aus dieser Klasse verwendet werden.
Möchten Sie nun abfragen, ob z.B. die Taste F5 auf der Tastatur gedrückt wurde, so schreiben Sie dies so in den Code:
\begin{lstlisting}{Input}
win->GetInput().CheckKey(KEY_F5);
\end{lstlisting}
Möchten Sie nun überprüfen, ob die Linke Maustaste gedrückt wurde, so tun Sie dies folgendermaßen:
\begin{lstlisting}{Input}
win->GetInput().CheckMouse(MOUSE_LEFT);
\end{lstlisting}

Möchte man nun die Position des letzten Klicks der linken Maustaste ermitteln, benutzt man diese Methode:
\begin{lstlisting}{Input}
win->GetInput().GetLastposClick(MOUSE_LEFT);
\end{lstlisting}
Diese gibt einen zweidimensionalen Vektor vom Typ Float zurück, welcher die Position vom letzten Klick der Maus mit einer bestimmten Maustaste beschreibt. 
Die folgende Methode gibt Daten in Form eines zweidimensionalen Vektors mit der aktuellen Position der Maus zurück:
\begin{lstlisting}{Input}
win->GetInput().GetLastposInst();
\end{lstlisting}

Alle Tasten der Tastatur und Maus können über Makros angesprochen werden. Auch können Sie die ASCII-Codes der einzelnen Tasten verwenden. Die Makros, welche die Tasten der Tastatur beschreiben, starten mit \textit{"`KEY\_"'}. Die Makros, die für die Maus verwendet werden, starten mit \textit{"`MAUS\_"'}.
(Für weitere Informationen: Siehe DoxyGen Dokumentation)