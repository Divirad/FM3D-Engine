\subsection{Animation}

Für einer lebendige Szene in einem Spiel reicht es nicht aus nur statische Models zu verschieben und zu drehen, man muss auch die Models an sich ändern können. Die FM3D-Engine verwendet eine Animationsart namens \textit{Skeletal animation}. Hierbei bekommt jedes animierte Mesh ein Skelett zugewiesen, wobei sich mehrere verschiedene Meshes das gleiche Skelett teilen können. Zum Beispiel könnte es in einem Spiel ein Mesh für einen Ritter und eins für einen Bauern geben. Diese sehen unterschiedlich aus, aber beide könnten das gleiche Skelett haben und somit die gleichen Animationen.

Ein Skelett besteht aus mehreren Knochen, die jeweils weitere Knochen besitzen. Wenn sich ein Knochen bewegt, bewegt er alle an ihm hängende Knochen und die wiederum alle an ihnen hängenden. Somit wenn sich der Knochen des linken Armes bewegt, bewegen sich auch gleichzeitig alle Fingerknochen. Jeder Knochen besitzt unabhängig von den anderen eine Ausgangsposition und eine Id. Die Id wird verwendet um herauszufinden welcher Knochen sich auf welchen Vertex auswirkt und welche Knochen gar nicht verwendet werden. Jeder Vertex kann von bis zu 4 Knochen transformiert werden. Dazu besitzt jeder Vertex 4 Knochen-Ids und 4 floats die angeben wie stark sich ein Knochen auf den Vertex auswirkt wie in \cref{table:VertexAufbau} zu sehen ist. Diese Daten werden in einem externen Programm beim Erstellen des Mesh festgelegt und der Vorgang wird \textit{Weight painting} genannt. Ein Beispiel aus dem Programm Blender ist in \cref{Img:Skeleton} zu sehen.

Eine Animation besitzt zu verschiedenen Zeitpunkten eine Position, Rotation und Skalierung für einen Knochen genannt \textit{Keyframe}. Dabei ist es nicht nötig, dass alle Knochen die gleiche Anzahl an Keyframes haben oder einen Keyframe an der gleichen Zeitposition. Es ist auch möglich, dass Keyframes nicht Position, Rotation und Skalierung auf einmal beinhalten sondern nur eins oder zwei davon. Um den Zustand aller Knochen zu einem bestimmten Zeitpunkt herauszufinden fängt man am obersten Knochen der Hierarchie an und arbeitet sich dann schrittweise nach unten, da die unteren Knochen von den oberen abhängig sind. Falls der Zustand zu diesem Zeitpunkt nicht zufällig durch die Keyframes genau definiert ist, muss zwischen den zwei am nächsten liegenden Keyframes interpoliert werden.

Für die Position und Skalierung kann zwischen den Keyframes linear interpoliert werden:

$t_{0} \leq t \leq t_{1}$ \qquad $t_{0} < t_{1}$ \qquad Keyframes: $\overrightarrow{P_{0}}, \overrightarrow{P_{1}}$

$f = \dfrac{t - t_{0}}{t_{1} - t_{0}}$

$\overrightarrow{P} = ((\overrightarrow{P_{0}} \cdot (1 - f)) + (\overrightarrow{P_{1}} \cdot f)$

Für die Rotation muss \ac{SLERP}\cite{WikiSlerp} angewendet werden, dadurch bleibt die Kreisgeschwindigkeit über die Zeit konstant. Die Berechnung erfolgt für zwei Quaternionen, die eine Rotation repräsentieren.

$t_{0} \leq t \leq t_{1}$ \qquad $t_{0} < t_{1}$ \qquad Keyframes: $Q_{0}, Q_{1}$

$f = \dfrac{t - t_{0}}{t_{1} - t_{0}}$

$\phi = Q_{0} \cdot Q_{1}$

$\theta = \arccos(\phi) \cdot f$

$Q_{2} = Q_{1} - Q_{0} \cdot \phi$

$Q = Q_{0} \cdot \cos(\theta) + Q_{2} \cdot \sin(\theta)$

\begin{figure}
	\centering
	\begin{minipage}{0.49\textwidth}
		\includegraphics[width=\textwidth]{02theorie/skeleton.jpg}
	\end{minipage}
	\hfill
	\begin{minipage}{0.49\textwidth}
		\includegraphics[width=\textwidth]{02theorie/weightpainting.jpg}
	\end{minipage}

	
	Links: Skelett in Blender, Rechts: Weight painting eines Arms in Blender 
	
	Quelle: https://cgi.tutsplus.com/tutorials/building-a-basic-low-poly-character-rig-in-blender--cg-16955
	\caption{Blender Skelett}\label{Img:Skeleton}
\end{figure}