\section[Rendering]{Rendering\cite{GLReference, OglDev, ThinMatrix, SparkyEngine, GLTut}}
\label{rendering}
\subsection{OpenGL Grundlagen}

Die FM3D-Engine verwendet für die Darstellung von 3D-Szenen die Grafikbibliothek OpenGL. (Genaueres zu der Bibliothek erfährt man in \cref{opengl}). Mit OpenGL ist es möglich verschiedene kleine grafische Objekte zu rendern. Dieses entsprechen drei geometrischen Grundobjekte (\textit{Primitives}): Punkte, Linien und Dreiecke. Zudem gibt es verschiedene Möglichkeiten diese aneinander zu reihen, wie in \cref{OpenGLPrimitives} dargestellt ist. Sie können alle einzeln, aber auch aneinanderhängend gerendert werden, sodass Speicher bei den angrenzenden Eckpunkten gespart wird. Diese \textit{Primitives} werden durch ihre Eckpunkte oder auch \textit{Vertices} definiert. Diese beschreiben eine Position in einen dreidimensionalen Raum. OpenGL arbeitet mit einem orthografischem Koordinatensystem. Das heißt: Alle drei Achsen sind orthogonal zu einander und reichen von -1 bis 1.

\subsubsection{Frame buffer}
Das aktuelle Bild wird in mehreren Buffern gespeichert. Die Größe der Buffer ist direkt proportional zu der Pixelanzahl des \textit{Viewports}, also zu dem Bereich in welchem das gerenderte Bild dargestellt wird. Am wichtigsten ist der Color-Buffer. Dieser speichert die Farbe jedes Pixels in vier Floats. Jeweils einen für Rot, Grün, Blau und einen Alpha-Wert, welcher die Transparenz beschreibt. Des Weiteren gibt es den Depth-Buffer. Dieser ist selten auch als Z-Buffer in der Literatur zu finden. Er beschreibt den Abstand zwischen dem aktuell gerenderten Pixel und der Kameraebene. Diese Information wird als Farbinformation in dem Depth-Buffer gespeichert. 
Man benötigt nur einen Channel\todo{welcher Channel?}, da nur ein einziger Zahlenwert gespeichert werden muss. So ist, wenn man den Buffer anzeigt nur ein Schwarz-Weiß-Bild zu sehen. Je heller der Pixel ist, desto weiter weg befindet er sich. Die Größe des Depth-Buffer ist einstellbar. Je größer er ist, desto größer ist auch die Präzision, wobei der Depth-Buffer immer eine viel größere Präzision in der Nähe der Kamera besitzt und nach weiter hinten an Präzision verliert. Dies ist von Vorteil wenn Objekte sehr nah an der Kamera gerendert werden, da dort eine sehr hohe Präzision erforderlich ist. Weit weg von der Kamera ist die Präzision aber nicht sehr relevant. Der Depth-Buffer ist optional. Wird er nicht verwendet, so kann nicht die räumliche Anordnung der Primitives ermittelt werden. Welcher Pixel am Ende angezeigt wird, ist von der Reihenfolge, in der die Primitives gerendert werden, abhängig, wobei das zuletzt gerenderte Primitve ganz vorne zu sehen ist. Diese Buffer sind in \cref{DepthBuffer} dargestellt. 

Zudem gibt es noch den \textit{Stencil-Buffer}. Dieser ist auch optional und ordnet jedem Pixel einen bestimmten Wert zu. Er kann verwendet werden, um bei bestimmten Pixeln das Rendern zu verhindern, wie er dies umsetzt ist einstellbar. Es ist möglich verschiedene Operationen durchzuführen, wenn ein Pixel gerendert wird. Es ist zum Beispiel möglich, dass der Stencil-Wert auf 1 gesetzt oder er um 1 erhöht wird. Zudem ist es möglich den Stencil-Test einzustellen. Dieser wird bei jedem Renderprozess eines Pixels ausgeführt. So kann man zum Beispiel erreichen, dass nur dort, wo der Stencil-Buffer den Wert 1 besitzt ein Pixel gerendert wird. Dadurch können \textit{Schablonen} erstellt werden, bei welchen Objekte gerendert und bei welchen nicht gerendert werden können. daher auch der Name. \todo{Stencil buffer?} 
Alle Buffer zusammen werden in einem \ac{FBO} gespeichert. Dieses ermöglicht das gleichzeitige aktivieren aller zugehörigen Buffer, sowie das Anzeigen des Color-Buffers auf dem Bildschirm. Es muss nicht speziell ein \ac{FBO} erstellt werden. Wenn keines erstellt wird, so wird standardmäßig direkt auf den \textit{Screen-Buffer} gerendert. Ein weiterer Vorteil ist, dass man ein \ac{FBO} sowohl als Output (die häufigere Verwendung) als auch als Input verwenden kann, so kann man zum Beispiel eine gerenderte 3D-Szene in einer anderen 3D- oder 2D-Szene einfügen.  

\subsubsection{Buffer Objects}
Will man einen oder mehrere dieser \textit{Primitives} rendern, so laufen die Daten, welche diese \textit{Primitives} definieren, durch verschiedene Schritte. Diese werden in der Literatur auch oft allgemein als die Rendering-Pipeline bezeichnet. Bei Verwendung dieser Pipeline können immer nur eine Art von Primitives gleichzeitig gerendert werden können.
Die Daten werden dabei als Buffer auf der \ac{GPU} gespeichert, verwendet werden dazu die \acp{VBO}, welche als Byte-Arrays vorliegen. Bei ihnen muss manuell eingestellt werden, wie diese Bytes interpretiert werden sollen. Dafür beschreibt man verschiedene Attribute mit Byte-Anzahl und Größe sowie DatenTyp. Zum Beispiel beschreiben die ersten 4 Bytes einen Float und die darauffolgenden 12 einen 3D-Float-Vektor. Es können auch mehrere \ac{VBO}s zusammengefasst werden in einen \ac{VAO}. \acp{VAO} speichern zusätzlich den Zustand der enthaltenen \acp{VBO} und die Information welche Attribute verwendet werden. Zusätzlich zu einem oder mehreren \acp{VBO}, kann ein \ac{IBO} verwendet werden. Dieser besteht aus einem Array von Ganzzahlen und wird ebenfalls auf der \ac{GPU} gespeichert. Er beschreibt welche Vertices für welche Primitives verwendet werden sollen, wo bei Vertices mehrfach verwendet werden können. Der Index-Buffer wird nacheinander durchgegangen und jeder darin gespeicherte Index steht für einen Wert im \ac{VBO}. Wenn man also viele Primitives hat, die sich einen gemeinsamen Punkt teilen, ist es effizienter einen Index-Buffer zu verwenden, da dieser geimeinsame Vertex nur einmal gepseichert werden muss und in der Regel ist die Größe eines Vertex weitaus größer als die Größe eines Index. Zum Beispiel können mit dem \ac{IBO} { 0, 1, 2, 2, 1, 3} zwei Dreiecke gerendert werden, es werden aber nur 4 Vertices benötig (0-3).\cite{ThinMatrix}

\subsubsection{Pipeline\cite{Pipeline}}
Ein \ac{VBO} oder ein \ac{VAO} entsprechen dem Input der Pipeline, wobei in modernen Programmen immer \acp{VAO} verwendet werden, da sie zusätzliche Informationen speichern können.

In der Pipeline bestehen mehrere Schritte aus Shadern. Dies sind kleinere Programme, welche auf der Grafikkarte ausgeführt werden. Sie werden in der Programmiersprache GLSL programmiert. Dies eine spezielle Sprache, die für Shader von OpenGL verwendet wird. Sie ähnelt stark C und besitzt einige bereits zur Verfügung gestellte Funktionen um linear algebraische Rechnungen durchzuführen. Sie unterscheidet sich stark zwischen den einzelnen OpenGL Versionen, da ständig neue Features hinzugefügt werden. Die FM3D-Engine besitzt daher auch teilweise für verschiedene OpenGL-Versionen verschiedene Shaderimplementationen.

Ein grober Überblick über die Rendering Pipeline wird in \cref{RenderingPipeling} gegeben. Der erste Schritt ist das Erstellen der Daten und das Angeben der Attribute, wie vorher beschrieben. Diese sind dann der Input für den \textit{Vertex-Shader}. 

Im \textit{Vertex-Shader} werden die Positionen einzelner Vertices festgelegt, daher wird er für jeden \textit{Vertex} einmal ausgeführt. Hier können zum Beispiel Operationen wie Verschiebungen durchgeführt werden um alle Eckpunkte zu verschieben. Der einfachste mögliche Vertex-Shader liest eine Position aus dem \ac{VBO} und verwendet sie als Vertexposition. 

Der Ausgang des Vertex-Shader ist der Eingang des \textit{Tessellation-Shader}. Dieser ist optional und erst seit OpenGL-Version 4.0 verfügbar. Er wird in der FM3D-Engine nicht verwendet, daher gehen wir nicht weiter auf ihn ein. 

Als nächster Schritt folgt der \textit{Geometry-Shader}. Er wird für jedes Primitive einmal ausgeführt und bekommt dieses auch als Eingabe. Zusätzlich erhält er noch die Ausgabe des Tessellation Shader bzw. des Vertex Shader. Hier können Operationen ausgeführt werden, die für jedes Primitive ausgeführt werden müssen. Es ist auch möglich die Art des Primitive zu ändern: zum Beispiel könnte man aus einem Dreieck drei Linien machen. Der Geometry-Shader ist ebenfalls optional.

Nach diesen 3 Shadern werden einige nicht veränderbare Operationen mit den Daten durchgeführt, auch bekannt als \textit{fixed function processing}. Primitives die sich nicht mehr in dem Bereich von -1 bis 1 befinden werden entfernt und nicht gerendert, Primitives die sich genau auf der Grenze befinden werden geteilt, so dass nur der Teil im erlaubten Bereich gerendert wird. Dieses Verfahren bezeichent man als Clipping.

Der nächste Schritt ist abhängig von der Art von Primitive die man gewählt hat. Wenn man eine zusammenhängende Reihe von Primitives rendert wird diese hier aufgelöst und in einzelne Primitves aufgeteilt, so dass folgende Operationen immer auf ganze Primitves ausgeführt werden und nicht nur auf einen stream von Vertices. Hier wird auch eine Operation names \textit{Face Culling} durchgeführt. Diese wird nur für Dreiecke durchgeführt. Alle Dreiecke, die von der Kamera weg zeigen, werden weggelassen ohne sie zu rendern, so kann verhindert werden, dass bei Objekten die komplett aus Dreiecken umschlossen sind, wo die Rückseite eines Dreiecks also sowieso nie zu sehen ist, keine unnötigen Dreiecke gerendert werden. Es kann eingestellt werden ob diese Operation durchgeführt werden soll oder nicht. Zum Beispiel für transparente Objekte kann sie nicht verwendet werden, da dort die Rückseite trotzdem zu sehen ist.

Danach folgt der Schritt namens \textit{Rasterization}. Er beschreibt die Umwandlung von Primitives in \textit{Fragments}, also Pixel auf dem Bildschirm. Hier werden weitere Optimisierungsvorgänge durchgeführt um keine unnötigen \textit{Fragments} zu rendern.

Der vorletzte Schritt ist wieder ein Shaderprogramm. Der \textit{Fragment-Shader} wird für jeden Fragment einmal ausgeführt und bestimmt die Farbe dieses Pixels. Als Input bekommt er den Output des davor ausgeführten Shaders, wobei die Werte für jeden Fragment interpoliert werden und so eine Mischung aus den Daten jedes Vertex generiert wird. Diese verteilen sich linear über das Primitive. Der Fragment-Shader ist der mit Abstand am häufigsten ausgeführte Shader, daher sollten alle nicht unbedingt benötigten Berechnungen in vorherigen Shadern ausgeführt werden.

Als finalen Schritt werden verschiedene Tests für den Fragment ausgeführt wie Depth- und Stencil-Test um zu bestimmen ob das Fragment wirklich angezeigt werden soll. Wenn alle Tests positiv sind, wird das Fragment mit dem bestehenden Fragment auf dem Buffer verrechnet.

\begin{figure}
	\centering
	\includegraphics[scale=0.7]{02theorie/openglPrimitives.png}
	\includegraphics[scale=0.7]{02theorie/openglPrimitives2.png}
	
	
	Quelle: http://www.lighthouse3d.com/tutorials/glsl-tutorial/primitive-assembly/
	\caption{OpenGL Primitives}\label{OpenGLPrimitives}
\end{figure}


\begin{figure}
	\centering
	\includegraphics[scale=0.5]{02theorie/DepthBuffer.jpg}
	
	
	Oben: Color, Unten: Depth
	
	Quelle: https://de.wikipedia.org/wiki/Datei:Z-buffer\textunderscore no\textunderscore text.jpg
	\caption{Color- und Depth-Buffer}\label{DepthBuffer}
\end{figure}


\begin{figure}
	\centering
	\includegraphics[scale=0.4]{02theorie/RenderingPipeline.png}
	
	
	Quelle: https://www.khronos.org/opengl/wiki/Rendering\textunderscore Pipeline\textunderscore Overview
	\caption{OpenGL Primitives}\label{RenderingPipeling}
\end{figure}
\subsection{Physically Based Rendering}

In einer 3D-Szene eines Videospiels will man natürlich komplexere Objekte als simple Primitives darstellen. Diese Objekte bestehen aus vielen Primitives, in diesem Falle sind meist Dreiecke. 
Die Art der Zusammensetzung ist unterschiedlich: manchmal kann es effizienter sein aneinander hängende Dreiecke zu verwenden. Jedoch werden meistens nur einzelne Dreiecke gerendert. 
Einige Programme verwenden %auch andere 
Polygone, welche aber auch aus Dreiecken bestehen. Daher ist dies irrelevant. 
Ein Objekt, gebildet aus Dreiecken, nennt man \textit{Mesh}. Als Beispiel ist das Mesh eines Delfins in \cref{Dolphin} dargestellt. Die Farbe bzw. die "`Haut"' eines Objektes wird aus einem zweidimensionalem Bild gelesen. Genannt wird dies eine \textit{Textur}.
 
Dafür besitzt jeder Vertex zusätzlich zu seiner Position eine zweidimensionale Position auf der Textur. So kann ermittelt werden, welche Pixel der Textur verwendet werden sollen.
Mit Mesh und Textur ist es möglich ein Objekt in einer dreidimensionalen Szene anzuzeigen. In einem Videospiel würde dies jedoch nicht überzeugen. Es wird \ac{PBR} verwendet. \ac{PBR} ist eine Möglichkeit eine realistischere dreidimensionale Szene zu erzeugen, in dem physikalische Phänomene, wie Licht, berücksichtigt werden. 
Wie ein Objekt auf Licht reagiert, hängt von verschiedenen Größen ab, wie die eigentliche Farbe des Objekts, die Oberflächenstruktur, aber auch die Farbe und Richtung des einfallenden Lichts. Alle Eigenschaften eines Objekts, die sich auf die Farbe auswirken, werden in einem \textit{Material} zusammengefasst. 
Ein Objekt, welches gerendert werden soll, besitzt beides: 
ein \textit{Mesh} und ein \textit{Material}. Das ganze wird zusammengefasst \textit{Model} genannt.

Es gibt zwei Arten von Lichtquellen in der FM3D-Engine: \textit{Directional Light} und \textit{Point Lights}. Ein Directional Light besitzt eine Richtung, aber keine Position. Diese Art kann verwendet werden, um zum Beispiel eine Sonne zu simulieren, die ist soweit von der Erde entfernt, dass die Position irrelevant wird. Die Richtung der Strahlen hingegen ist wichtig und von der Tageszeit abhängig. 

Point Lights sind das genaue Gegenteil. Sie besitzen keine Lichtrichtung, sondern scheinen in alle Richtungen gleich. Dafür besitzen sie eine genau festgelegte Position, die relevant relevant ist, da die Lichtstärke mit zunehmendem Abstand kleiner wird. Point Lights können verwendet werden, um die meisten Lichtquellen darzustellen (als Beispiel: Laternen oder Fackeln).

Man kann aber nicht nur zwischen Lichtquellen unterscheiden, sondern auch zwischen verschiedenen Arten des ausgesendeten Lichts. Die \ac{FM3D}-Engine verwendet ein Lichtmodell genannt "`Ambient/Diffuse/Specular"'. 
\textit{Ambient Light} ist das Licht, dass man jeden Tag sieht, auch wenn gerade keine Sonne scheint oder man sich nicht in direkter Nähe einer Lichtquelle befindet. 
Es entsteht dadurch, dass Licht von allen Objekten wieder teilweise reflektiert wird und so eine schwache und gleichmäßige Beleuchtung entsteht. Ohne das Ambient Light wäre es hinter einem Haus, welches die Sonne verdeckt, komplett finster. 

Die zweite Lichtart ist \textit{Diffuse Light}, welches abhängig von dem Auftrittswinkel des Lichtstrahls ist. Betrachten wir einen Würfel, so sehen wir: 
die dem Licht zugewandte Seite ist heller, als die nur teilweise zugewandte Seite und die abgewandte Seite erfährt gar kein \textit{Diffuse Light}. 

\textit{Specular light} modelliert die Lichtstrahlen, welche von einem Objekt reflektiert und in die Linse der Kamera bzw. in die Augen des Betrachters gelangen. Dies wird als blendendes, helles Licht wahrgenommen und ist oft auf metallischen Obeflächen zu erkennen. Diese drei Lichtarten sind in \cref{Img:Lights} dargestellt.

Alle Lichtberechnungen müssen für jeden Pixel ausgeführt werden und laufen daher im Fragment-Shader ab.
Um die Farbe eines gerenderten Pixels zu bestimmen, benötigt man einige Informationen: die Position, die Farbe, den Normalenvektor und Specular-Factor des Pixels. Diese vier Informationen reichen für alle Lichtberechnungen aus. Die Phase, in der sie erstellt bzw. berechnet werden, ist unterschiedlich. Ein Teil wird außerhalb des Programms in externen Programmen erstellt und als Vertex im Mesh oder als Information im Material gespeichert. Dabei unterscheidet man zwischen:
\begin{itemize}
	\item Der Information, die sich zwischen verschiedenen Objekten unterscheidet, aber nicht innerhalb des Objekts
	\item Information, die für jeden Vertex anders ist aber für jeden Pixel nur linear interpoliert werden muss
	\item Information die für jeden Pixel anders ist
\end{itemize}

Erstere kann einfach im Material gespeichert werden, da jedes Objekt ein eigenes Material haben kann und es anders als ein Mesh nicht viel Speicher benötigt. Die Vertexinformationen werden im \ac{VBO} des Mesh gespeichert und automatisch linear interpoliert, wenn sie an den Fragment-Shader weiter gegeben werden. Pixelinformationen müssen einzeln in einer Textur gespeichert werden. Diese sind dann im Material enthalten, wobei diese nur referenziert werden, damit verschiedene Materials die gleichen Texturen verwenden können. In den Vertex-Informationen müssen hierfür zusätzlich Textur-Koordinaten gespeichert werden. Daraus ergeben sich die folgenden Werte, die in \cref{table:VertexAufbau} zu sehen sind.

\begin{table}
	\caption{Vertex Aufbau}
	\label{table:VertexAufbau}
	\centering
	\begin{tabular}{lll}\toprule[1.5pt]
		Datentyp & Name & Beschreibung \\\midrule
		3D-Vektor & Position & Position des Vertex \\
		2D-Vektor & Textur Koordinate & Position des Vertex auf der Textur \\
		3D-Vektor & Normal & Normalenvektor zum Dreieck \\
		32-Bit Farbe & Color & Farbe des Vertex (optional, Standard ist weiß) \\
		3D-Vektor & Tangent & Tangentenvektor des Dreiecks. Benötigt wenn \\
		 & & Normal maps verwendet werden. Siehe \cref{section:Normalmapping}\\\bottomrule[1.5pt]
	\end{tabular}
\end{table}
\begin{table}
	\caption{Material Aufbau}
	\centering
	\begin{tabular}{lll}\toprule[1.5pt]
	Datentyp & Name & Beschreibung \\\midrule
	32-Bit Farbe & Color & Farbe des gesamten Objekts \\
	Textur & Color Texture & Gibt die Farbe jedes Pixels des Objektes an \\
	Textur & Normal map & Gibt den Normalenvektor jedes Pixels an. \\
	 & & Genaueres in \cref{section:Normalmapping} \\
	Float & Specular factor & Faktor für das resultierende Specular light \\
	Textur & Specular map & Specular factor für jeden Pixel. Der Faktor \\
	 & & des ganzen Objekts wird weiterhin verwendet \\
	 Boolean & UseWireframe & Gibt an ob ganze Dreiecke gerendert werden sollen\\
	  & & oder nur die Kanten. (Nützlich für Debugging)\\\bottomrule[1.5pt]
\end{tabular}
\end{table}


\begin{figure}
	\begin{center}
		\includegraphics[width=0.5\textwidth]{06anhang/bilder/delphin.jpg}
		\caption{Mesh eines Delfins}
		\label{Dolphin}
	\end{center}
\end{figure}
\todo[inline]{delfin verschieben}
\begin{figure}
	\centering
	\includegraphics[scale=0.4]{02theorie/amb_diff_spec.png}
		
	Quelle: https://clara.io/img/pub/amb\textunderscore diff\textunderscore spec.png
	\caption{Lights}\label{Img:Lights}
\end{figure}

Die FM3D-Engine verwendet \textit{Deferred Rendering}. Dies bedeutet, dass zuerst alle Objekte einer Szene gerendert und die Ergebnisse daraus in mehreren Buffern gespeichert werden. Danach wird für jede Lichtquelle ein weiterer Renderprozess ausgeführt, wobei die vorher genannten Buffer als Input dienen. Das Ergebnis ist in der \cref{Img:Lights} zu sehen. Der Vorteil hierbei ist, dass keine unnötige Lichtberechnungen ausgeführt werden muss, da alle Pixel später zu sehen sind und es keine Begrenzung für die Anzahl der Lichtquellen gibt. 
