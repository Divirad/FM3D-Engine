\section{Engine - Allgemeiner Aufbau}

Die Engine ist eine Visual C++ Laufzeitbibliothek (Dynamic-Link-Library; DLL). Diese DLL enthält den gesamten Code, welcher zum Erstellen eines Spieles erforderlich ist. Jason Gregory beschreibt in seinem Buch den Aufbau einer Game-Engine als "'\textit{eine Struktur aus Systemen und Schichten}"`. 
Hierbei können mehrere Schichten übereinander liegen. Ideal verwenden die höher gelegenen Schichten bereitgestellte Funktionalitäten der unteren Schichten, was aber nie umgekehrt geschehen soll. Dies ermöglicht die Abschirmung von hardwareabhängigen und spielnahen Klassen. Ein System übernimmt eine bestimmte Aufgabe und kann sich über mehrere Schichten ausbreiten. \cite{gea}
Die FM3D-Klasse hat diese Idee in bestimmten Teilen übernommen. Im Code ist die System-Struktur als Ordner-Struktur wiederzufinden. Diese Systeme sind teilweise auch durch einen \textit{Namespace} oder Präfix vor jedem Namen zu erkennen. Das Schichtenmodell ist in der Engine als Untersysteme (auch Subsystem genannt) erkennbar. Diese Untersysteme sind genauso zu erkennen und verhalten sich wie normale Systeme. Der einzige Unterschied ist, dass sie sich bereits in einem System befinden.