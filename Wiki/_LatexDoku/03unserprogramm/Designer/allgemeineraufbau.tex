\section{Designer - Allgemeiner Aufbau}
\label{designer}
Der \textit{Designer} ist ein Tool, um die Programmierung mit der Engine zu erleichtern. Er generiert über eine sogenannte \textit{Pipe}, die zu der Kommunikation zwischen dem Designer und der Entwicklungsumgebung \textit{VisualStudio2015} dient, um Code auf den Einstellungen des Users basierend zu generieren. Der Designer ist fähig ein volles funktionsfähiges Grundgerüst für ein Spiel zu generieren. So wird dem Nutzer des Designers viel Schreibarbeit gespart und er kann sich so den gewünschten Funktionen seines Spieles widmen.
Dieses eben genannte Grundgerüst besteht aus renderbaren Scenen mit den zugehörigen Entities\ref{entitysystem}, welche mit der \textit{Pipe} in ein VisualStudio-Projekt hinzugefügt werden.
Die Verwendung des Designers wird in dem folgenden Kapitel \ref{verwendung_designer} erklärt. Es wird im folgenden nur auf den Programmaufbau eingegangen.
Mit dem Designer können alle Projekte zwischengespeichert werden. Sowohl die erstellten Projektdateien mit den Pfaden zu hinzugefügten Dateien und den Einstellungen, aber auch die verschiedenen Scenen und Entity-Presets können gesichert werden. Dies hat den Vorteil, dass man Entities oder Scenen aus alten Projekten einfach in neuen verwenden kann. Zudem werden die Texturen für Modelle mit dem Designer in ein Format konvertiert, welches für die Engine einfacher zu verarbeiten ist, als herkömmliche Bilddateien.
Der Designer arbeitet mit eigens für ihn entwickelten Dateiformaten. Die meisten sind in Form einer .xml Datei (Extensible Markup Language \cref{extensiblemarkuplanguage}) 
auffindbar, dennoch werden verschiedene Endungen verwendet, um unterschiedliche Datentypen von anderen zu trennen. Im folgenden werden einzelne ausschlaggebende Funktionen erklärt.