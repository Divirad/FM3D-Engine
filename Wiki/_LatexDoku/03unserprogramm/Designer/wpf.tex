\subsection{Windows Presentation Foundation}
\label{wpf}

Windows Presentation Foundation \ac{WPF} ist eine 2006 eingeführte Bibliothek. \cite{wikipedia_wpf} "`Sie vereint die Vorteile von DirectX, Windows Forms, Adobe Flash, HTML und CSS."'\cite{eiwpf} 
Das Aussehen der Anwendungen werden mit der \textit{Extensible Application Markup Language} \ac{XAML} deklarativ beschrieben. Die Logik wird mittels C-Sharp Implementiert. So sind Arbeitsschritte besser zu unterteilen und strukturieren. Zudem bieten C-Sharp und WPF einen übersichtlichen Syntax zu Programmierung der GUI-Komponente. 
Da die FM3DExtension \ref{extension}, die wir über Pipe \textit{"`fernsteuern"'} um den Code für die Entity-Presets %\ref{entitypresets} 
zu generieren, auch mit C-Sharp geschrieben wurde. Des Weiteren laufen die Bibliotheken für die GUI-Bibliotheken nur über C-Sharp und WPF. Zwar ist der Rechenaufwand aufgrund des enthaltenen bi-direktionalen \textit{Beobachters} relativ höher, dennoch ist dies für eine so kleine Anwendung nicht bedeutungsvoll. Ein \textit{Beobachter} ist in der Softwareentwicklung Entwurfsmuster, welches zu der Weitergabe von Änderungen an einem Objekt an von diesem Objekt abhängige Strukturen dient.