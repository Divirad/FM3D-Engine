\subsection{Windows Presentation Foundation}
\label{wpf}

Windows Presentation Foundation \ac{WPF} ist eine 2006 eingeführte Bibliothek. \cite{wikipedia_wpf} "`Sie vereint die Vorteile von DirectX, Windows Forms, Adobe Flash, HTML und CSS."'\cite{eiwpf} 
Das Aussehen der Anwendungen werden mit der \textit{Extensible Application Markup Language} \ac{XAML} deklarativ beschrieben. Die Logik wird mittels C\# implementiert. So sind Arbeitsschritte besser zu unterteilen und zu strukturieren. Zudem bieten C\# und WPF einen übersichtlichen Syntax zur Programmierung der GUI-Komponente. 
Auch die VisualStudio Extension wurde in C\# geschrieben.
Des Weiteren sind die GUI-Bibliotheken nur in C\# und WPF implementierbar. Zwar ist der Rechenaufwand aufgrund des enthaltenen bi-direktionalen \textit{Beobachters} relativ höher, dennoch ist dies für eine so kleine Anwendung nicht bedeutungsvoll. Ein \textit{Beobachter} ist in der Softwareentwicklung Entwurfsmuster, welches zu der Weitergabe von Änderungen an einem Objekt an von diesem Objekt abhängige Strukturen dient.