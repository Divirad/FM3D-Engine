\subsection{Video-Spiele}

In der Videospielindustrie ist ein Spiel eine Software die Bilder oder virtuelle dreidimensionale Objekte, durch Spieler-Eingaben beeinflusst, um ein vom Entwickler (oder genauer „vom Spieldesigner“) festgelegtes Ziel zu erreichen. 
Im „Normalfall“ wird vom Spieler ein Menschen-, Tier-ähnliches Wesen oder Fahrzeug durch ein Eingabegerät fortbewegt. Betont wird hier der „Normalfall“, da es verschiedene Genres oder Video-Spiel-Formen gibt. Hier ist die Kreativität des Entwicklers gefragt.

Der Autor des Buches „Game Engine Architecture“ \cite{gea} definiert ein Video-Spiel aus technischer Sicht als „soft real-time interactive agent-based computer simulations“ (interaktive agentbasierte computergestützte Echtzeitsimulationen).
Doch wie kommt man zu diesem Begriff?

Die meisten Spiele bilden eine Simulation einer realen oder fiktiven Welt ab, die mathematisch modelliert wird. In einer Agent basierten Simulation interagieren verschiedene „Entities“ miteinander, weshalb auch objektorientierte Programmiersprachen (wie zB. C++ oder Java) verwendet werden.
Solche interaktive Videospiele sind zeitlich abhängige Simulationen, da sich Eigenschaften wie zum Beispiel das Aussehen in dieser fiktiven Welt mit der Zeit verändern. Außerdem sollte das Programm in Echtzeit auf die Eingaben des Spielers reagieren.
