\subsection{Video-Spiele}

In der Videospielindustrie ist ein Spiel eine Software, die Bilder oder virtuelle dreidimensionale Objekte, die durch Spieler-Eingaben beeinflusst werden können, um ein vom Entwickler (oder genauer vom Spieldesigner) festgelegtes Ziel zu erreichen. 
Im Normalfall wird vom Spieler ein menschen- oder tierähnliches Wesen oder Fahrzeug durch ein Eingabegerät fortbewegt. Betont wird hier der Normalfall, da es verschiedene Genres oder Videospiel-Formen gibt. Hier ist die Kreativität des Entwicklers gefragt.

Gregory definiert in dem Buch „Game Engine Architecture“ ein Video-Spiel aus technischer Sicht als „soft real-time interactive agent-based computer simulations“ \cite{gea}(interaktive agentbasierte computergestützte Echtzeitsimulationen).
Doch wie kommt man zu diesem Begriff?

Die meisten Spiele bilden eine Simulation einer realen oder fiktiven Welt ab, die mathematisch modelliert wird. In einer Agent basierten Simulation interagieren verschiedene „Entities“ miteinander, weshalb auch objektorientierte Programmiersprachen (wie z.B. C++ oder Java) verwendet werden.
Solche interaktiven Videospiele sind zeitlich abhängige Simulationen, da sich Eigenschaften, wie zum Beispiel das Aussehen, in dieser fiktiven Welt mit der Zeit verändern. Außerdem sollte das Programm in Echtzeit auf die Eingaben des Spielers reagieren.
