\subsection{Vergangenheit bis heute}
\label{vergangenheit}
Die ersten Konsolen auf dem Markt besaßen geringen Speicher. So mussten die Entwickler die Spiele immer speicherkalkuliert entwickeln, sodass sie möglichst klein und speichersparend waren.\cite{gbdetails} Da man also für jedes Spiel einen optimierten Code benötigte, waren Game-Engines von Drittanbietern nicht nötig.

Erst mit den 3D-Spielen und verbesserter Hardware wurden die Game-Engines von externen Unternehmen populär. Konsolen und PCs verfügten nun über genug Speicher und man musste nicht mehr auf jedes Byte achten. Die "`Angst"', man könne Speicher verschwenden oder ein zu großes Programm entwickeln, wurde mit der Zeit immer mehr zurück gedrängt.
Da die Entwickler nun auch Spiele herausbringen wollten, welche auf dem Markt besser ankommen und technisch fortschrittlicher sind, wurde der Code der Software um einiges komplexer.

So kam man schnell auf die Idee, ein System zu entwickeln, das für die Hauptbereiche eines Spiels ein Grundgerüst baut, die Software in Betrieb hält und das immer wieder verwendbar ist. Mit einer Game-Engine konnte man nun Zeit und Aufwand beim Programmieren sparen, mehr Zeit in die Optimierung des Spieles stecken und sich mehr Zeit für die Spielinhalte nehmen. 
