\subsection{Spiele}	

Bevor man eine Spiele-Engine programmiert, so sollte man sich erst einmal die Frage stellen, was genau ein \textit{Spiel} ist. Wir alle können durch unseren Alltag etwas mit dem Begriff „Spiel“ verbinden. Doch müssen wir für das allgemeine Verständnis auch hier noch einmal fest definieren, was genau ein Spiel ist;
Brettspiele wie Schach, Dame, Mühle; Glücksspiele wie Poker, Roulette, Lotto; Kinderspiele wie Topfschlagen; und natürlich die Video-Spiele. 
Dies sind für uns alle Spiele, doch warum?
Schlägt man den Begriff „Spiele“ in dem Nachschlagewerk „Der Brockhaus“ nach, so bekommt man die folgende Definition:

\begin{quote}
	„Spiel das, 1) jede Tätigkeit, die aus Freude an dieser selbst geschieht, im Ggs. zur zweckbestimmten Arbeit. […]“ (Aus \cite{brockhaus})
\end{quote}

Spiele dienen also der Unterhaltung und dem Zeitvertreib des Spielers.
Diese Beschreibung trifft auch auf Videospiele zu, denn auch Videospiele sind größtenteils nicht zweckbestimmt und dienen nur zur Unterhaltung des Spielers.
In dem Buch „Game Engine Architecture“ \cite{gea} ist ein Verweis auf das Buch „A Theory
of Fun for Game Design“ in dem Raph Koster den Begriff „Spiel“ wie folgt definiert:

\begin{quote}
	„Ein Spiel ist eine interaktive Erfahrung, die dem Spieler ermöglicht, steigende Herausforderungen von Mustern, welche der Spieler im Laufe des Spieles lernt, zu meistern.“
	(\cite{theoryoffun}, Zitat aus dem Englischen übersetzt)
\end{quote}