\subsection{Unity3D - Unity Technologies}

"`Unity3D"' ist eine Game-Engine der Firma \textit{Unity Technologies}, welche erstmals im Jahre 2005 veröffentlicht wurde. Sie ist für die 3D-Spiele Entwicklung konzipiert, besitzt eine eigene 2D-Engine, ist weitaus komplexer als der "`GameMaker"' und bietet somit auch mehr Funktionen. 
Mit ihr kann man Spiele für Computer, aber auch Spielkonsolen, mobile Geräte und Webbrowser entwickeln. Sie ist für Windows, Linux (nur Beta) und OSX in vier verschiedenen Versionen erhältlich, unter denen sich auch eine kostenlose Version befindet. Das Programm beinhaltet zudem noch einen 3D-Terrain-Modellierer, Baum- und Pflanzen-Modell-Editor, Werkzeuge für Partikeleffekte und ein Tool für Bewegungssteuerung für Charaktere.

Die grafische Entwicklungsumgebung ähnelt sehr modernen 3D-Animationsprogrammen. Sie stellt im Hauptfenster eine Spiel-Szene dar und verfügt zudem über eine sogenannte \textit{Timeline}, in der der Zeitablauf des Spieles dargestellt wird. In Menüs kann man Parameter des Spieles verändern und sie für die Strukturierung des Spieles verwenden. 

Das Spiel-Scripting basiert auf dem \textit{Mono}-Compiler und bietet verschiedene Scriptsprachen, darunter auch C-Sharp und eine vom Entwicklerteam eigens entwickelte Sprache "`UnityScript"'. Sie ist hauptsächlich für \textit{Cross-Platform} Programme gedacht. Das heißt, dass alle Scripte sofort für verschiedene Geräte exportiert werden können, ohne davor den Code an das Gerät anpassen zu müssen.

Eine aktuelle Szene besteht aus \textit{Game-Objects}, denen man verschiedene Eigenschaften (Materialien, Klänge, physikalische Eigenschaften, Skripte) zuordnen kann. Dies haben wir auch bei unserem Designer mittels Komponenten realisiert (siehe \cref{designer} oder \cref{entitysystem}).

Aufgrund dieser Komplexität ist die "`Unity3D-Engine"' eher für Fortgeschrittene geeignet und braucht weitaus mehr Einarbeitungszeit als der "`GameMaker"', sie bietet jedoch auch mehr Möglichkeiten.
Bei der Auswahl einer geeigneten Engine muss man abwägen, ob eine einfach zu bedienende Engine auch allen erforderlichen Entwicklungsmöglichkeiten bietet.
