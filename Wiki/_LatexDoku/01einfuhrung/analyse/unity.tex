\subsection{Unity3D - Unity Technologies}

"'Unity3D"` ist eine Spiele-Engine der Firma \textit{Unity Technologies}, welche erstmals im Jahre 2005 veröffentlicht wurde. Sie ist für die 3D-Spiele Entwicklung konzipiert, besitzt eine eigene 2D-Engine, ist weitaus komplexer als der "'GameMaker"` und bietet auch somit mehr Funktionen. Mit ihr kann man Spiele für Computer aber auch Spielkonsolen, mobile Geräte und Webbrowser entwickeln und ist für Windows, Linux (nur Beta) und OSX in 4 verschiedenen Versionen erhältlich, unter denen sich auch hier eine kostenlose Version befindet. Das Programm kommt zudem noch mit einem 3D-Terrain-Modellierer, Baum- und Pflanzen-Modell-Editor, ein Werkzeuge für Partikeleffekte und ein Tool für Bewegungssteuerung für Charaktere.

Die grafische Entwicklungsumgebung ähnelt sehr modernen 3D-Animationsprogrammen. Sie stellt im Hauptfenster eine Spiel-Szene dar und verfügt zudem über eine sogenannte \textit{Timeline} in der der Zeitablauf des Spieles dargestellt wird. In Menüs kann man Parameter des Spieles verändern und sie für die Strukturierung des Spieles verwenden. 

Das Spiel-Skripting basiert auf dem \textit{Mono}-Compiler und bietet verschiedene Skriptsprachen, darunter auch C-Sharp und eine vom Entwicklerteam eigens entwickelte Sprache "'UnityScript"`. Sie ist hauptsächlich für \textit{Cross-Platform} Programme gedacht. Das heißt, dass alle Skripte sofort für verschiedene Geräte exportiert werden können ohne davor den Code an das Gerät anzupassen.

Eine aktuelle Szene besteht aus \textit{Game-Objects} denen man verschiedene Eigenschaften (Materialien, Klänge, physikalische Eigenschaften, Skripte) zuordnen kann. Dies haben wir auch bei unserem Designer versucht zu realisieren (Siehe \ref{designer} oder \ref{entitysystem}).

Aufgrund dieser Komplexität der Engine ist sie aber eher für Fortgeschrittene geeignet und braucht weitaus mehr Einarbeitungszeit als der GameMaker. Jedoch sind die Grenzen hier weitaus weiter gesetzt. Wohingegen man bei dem "'GameMaker"` bei der Entwicklungsmöglichkeiten doch ein wenig mehr begrenzter ist.
Daraus kann man schließen: Eine simpel gehaltene Engine die einfacher zu bedienen ist muss nicht immer eine Bessere sein.
