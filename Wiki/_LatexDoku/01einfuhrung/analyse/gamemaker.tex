\subsection{GameMaker - YoyoGames}

Die zuerst für 2D-Animationen konzipierte Game-Engine "'GameMaker"` (In den Anfängen \textit{Animo}) wurde von Overmars erstmals im Jahre 1999 veröffentlicht und ist nun als "'GameMaker:Studio"` bekannt. Zunächst hat man sich bei der Entwicklung sehr stark auf die 2D-Spiele Entwicklung konzentriert. Man kann mit ihr nun sogar 3D-Spiele entwickeln. 

Das Programm ist momentan in drei verschiedenen Versionen erhältlich (Stand 2017). Darunter auch eine kostenlose Version mit beschränkten Funktionen.
Die Benutzeroberfläche ist sehr simpel gehalten und ist so konzipiert, dass sogar ein Leihe den Programmablauf eines Spieles durch einfaches "Drag'n'Drop" von Schaltflächen schnell erstellen und strukturieren kann. Zudem besitzt der "'GameMaker"` eine eigene Skript-Sprache, welche einige der höheren Programmiersprachen als Vorbild nimmt.

Die mit der GameMaker-Engine entwickelten Spiele werden durch verschiedene \textit{Resourcen} unterteilt;
Sprites, Sounds, Backgrounds, Paths, Scripts, Fonts, Timelines, Objects und Rooms. 
Auch wir wollten dies so ähnlich in unserem Designer handhaben und haben uns von dem Prinzip der Resourcen inspirieren lassen. \ref{designer}

Die Spiele können für diverse gängigen Betriebssysteme (Windows, Android, iOS ua.) aber auch für populäre Spielekonsolen (PlayStation, XBox ua.) kompiliert werden. 
