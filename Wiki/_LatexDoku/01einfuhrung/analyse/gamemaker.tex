\subsection{GameMaker - YoyoGames}

Die zuerst für 2D-Animationen konzipierte Game-Engine "`GameMaker"' (In den Anfängen \textit{Animo}) wurde von Overmars erstmals im Jahre 1999 veröffentlicht und ist nun als "`GameMaker:Studio"' bekannt. Zunächst hat man sich bei der Entwicklung sehr stark auf die 2D-Spiele Entwicklung konzentriert. Mittlerweile kann man mit ihr sogar 3D-Spiele entwickeln. 

Das Programm ist momentan in drei verschiedenen Versionen erhältlich (Stand 2017). Darunter auch eine kostenlose Version mit eingeschränkten Funktionen.
Die Benutzer-Oberfläche ist sehr simpel gehalten und so konzipiert, dass sogar ein Laie den Programmablauf eines Spieles durch einfaches "`Drag'n'Drop"' von Schaltflächen schnell erstellen und strukturieren kann. Zudem besitzt der "`GameMaker"' eine eigene Skript-Sprache, welche einige der höheren Programmiersprachen (zum Beispiel Pascal, Java und C)  als Vorbild hat.

Die mit der "`GameMaker"'-Engine entwickelten Spiele werden mit verschiedene \textit{Resourcen} entwickelt;
Sprites, Sounds, Backgrounds, Paths, Scripts, Fonts, Timelines, Objects und Rooms. 
Auch wir wollten dies so ähnlich in unserem Designer handhaben und haben uns von dem Prinzip der Resourcen inspirieren lassen. (Siehe \cref{designer})

Die Spiele können für diverse gängigen Betriebssysteme (Windows, Android, iOS u.a.), aber auch für populäre Spielekonsolen (PlayStation, XBox u.a.) kompiliert werden. 
