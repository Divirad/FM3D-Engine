\subsection{Game-Engine}	
\label{engine}
Wenn man Spiele entwickeln möchte, so taucht früher oder später das Wort „Game-Engine“ auf. Einige Entwicklungsfirmen verweisen auch auf ihre Engine, um technisch besser da zu stehen und ihrem Spiel ein besseres Ansehen zu verschaffen. Heißt das aber, mit einer besseren Game-Engine kann man ein besseres Spiel entwickeln? Um dies herauszufinden, sollte man sich erst einmal anschauen, wie solche Engines funktionieren und aufgebaut sind.

Frühere Spiele wie „Tetris“, „PacMan“ oder „Space-Invaders“ wurden immer „from scratch“  entwickelt.\todo{Quelle?} Die Entwickler besaßen also so gut wie gar kein Gerüst, bevor sie dieses Spiel programmierten. Da die meisten Spiele aber einen ähnlichen Aufbau besitzen, fiel auf, dass man immer einen ähnlichen Programm-Code verwenden könnte, um die Grundstruktur eines Spieles zu entwickeln. Vereinfacht verfügt ein Spiel über die Zentralbereiche der Grafik und die der Logik. Die Soundausgabe ist bei Spielen optional, da auch Spiele ohne Musik und Sound-Effekte existieren. Dennoch bieten die meisten Engines auch hierfür ein Grundgerüst.

Eine Game-Engine bietet dem Entwickler also diese Rahmenfunktionen, die seiner Software das Grundgerüst liefert. Dies heißt in der Softwareentwicklung „Framework“. 
Moderne Game-Engines sind auf ein bestimmtes Genre oder auf bestimmte Aufgabenbereiche spezialisiert. So gibt es Game-Engines sowohl für 2D- als auch 3D-Spiele. Der „GameMaker“ von „YoyoGames“ ist zum Beispiel auf 2D-Spiele spezialisiert. Doch kann man auch Spiele in 3D mit ihm erstellen. Die „CryEngine“ hingegen ist hauptsächlich für 3D-Spiele konzipiert.\todo{Quelle?} (Für weitere Details zu diesem Thema siehe \ref{engines}.)
