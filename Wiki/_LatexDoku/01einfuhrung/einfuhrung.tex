
\chapter{Einführung}
\label{c:einführung}
\setcounter{page}{1}
\section[Warum dieses Themenfeld?]{Warum haben wir dieses Themenfeld gewählt?}

Uns ist Aufgefallen, dass man bei größeren Schulaufgaben, in denen man auf sich alleine gestellt ist, mehr mit einer Programmiersprache ausprobieren und experimentieren kann. Das Erfolgserlebnis beim Fertigstellen einer Software ist immens. 

Da ein Großteil der Schüler in der heutigen Zeit mit dem Medium „Videospiele“ konfrontiert ist und dieses perfekt zum Ausprobieren ist, kann es für sie ein höchst interessanter Prozess sein, ein eigenes Videospiel zu entwickeln. Dies haben wir auch praktisch in unserem eigenen Unterricht positiv zu spüren bekommen.
Leider ist die Entwicklung eines Spieles ein sehr komplexer Prozess, welcher bei großen Industrien sogar in vier bis fünf Arbeitsbereiche unterteilt wird; (Produzent), Programmierer, Grafiker, Spiele-Designer und Sounddesigner.  \cite{gea}

Für die Entwicklung eines Spieles ist leider einige Fachkompetenz von Nöten, die nicht jeder Schüler besitzt. Zudem ist die Entwicklung „from scratch“ eines Spieles sehr zeitaufwendig und nicht jeder Lehrer kann so viel Zeit für ein Projekt zur Verfügung zu stellen. 
So könnte man nun also eine Engine verwenden, um ein Spiel zu entwickeln. Hier kommt nun aber ein weiteres Problem auf; Die meisten Engines sind sehr kostspielig oder verfügen über eine eigene Programmier-/ Skriptsprache und sind rein zum Spiele-Entwickeln konzipiert, meist auch nur mit kommerzieller Basis, oder um Zeit zu sparen und nicht um die Struktur einer Programmiersprache zu erlernen.

Darum wurde die Flying-Mind 3D [FM3D] Engine entwickelt. Die FM3D Engine besitzt die wichtigsten Komponenten um ein Spiel zu entwickeln;  Ein Mathsystem, ein Memorysystem, ein Filesystem, ein Grafiksystem und einige Funktionen die es sehr leicht machen ein Spiel zu entwickeln.
Zudem wurde ein „Designer“ entwickelt in den man die 3D-Modelle laden, positionieren und die Ausgangssituation des Spiels zu einem Visual-Studio 2015 Projekt konvertieren kann.
Der Lehrer könnte dem Schüler zB. die Aufgabe geben, einen kleinen Roboter durch ein Labyrinth zu einer Fahne hindurch zu manövrieren. Der Schüler kann nun alle Einstellungen im Designer tätigen, sich ein funktionsfähiges Projekt generieren lassen und sich vollkommen auf die Programmierung der Steuerung des 3D-Modell des Roboters fokussieren.

Natürlich können die Schüler mit der Engine auch größere Projekte programmieren und auch für private Projekte gebrauchen. Auch ist die Engine nicht nur für Schüler entwickelt worden. Auch Hobby-Spieleentwickler können mit ihr einfach und simpel eigene Spiele entwickeln, ohne dabei den Überblick im Code zu verlieren.

Die FM3D-Engine kann auch für Programme mit dreidimensionalen Rendering verwendet werden, welche nicht in den Anwendungsbereich Spiele fallen. Darunter fallen zum Beispiel Lehrsoftware, Simulationen oder mehr.

\section{Begriffsdefinitionen}
\subsection{Spiele}	

Bevor man eine Spiele-Engine programmiert, so sollte man sich erst einmal die Frage stellen, was genau ein \textit{Spiel} ist. Wir alle können durch unseren Alltag etwas mit dem Begriff „Spiel“ verbinden, doch müssen wir für allgemeines Verständnis auch hier noch einmal fest definieren, was genau ein Spiel ist;
Brettspiele wie Schach, Dame, Mühle; Glücksspiele wie Poker, Roulette, Lotto; Kinderspiele wie Topfschlagen; und natürlich die Video-Spiele; Dies sind für uns alle Spiele, doch warum?
Schlägt man den Begriff „Spiele“ in dem Buch „Der Brockhaus“ nach, so bekommt man die folgende Definition:

\begin{quote}
	„Spiel das, 1) jede Tätigkeit, die aus Freude an dieser selbst geschieht, im Ggs. zur zweckbestimmten Arbeit. […]“ (Aus \cite{brockhaus})
\end{quote}

Spiele dienen also der Unterhaltung und dem Zeitvertreib des Spielers.
Diese Beschreibung trifft auch auf Videospiele zu, denn auch Videospiele sind größtenteils nicht zweckbestimmt und dienen nur zur Unterhaltung des Spielers.
In dem Buch „Game Engine Architecture“ \cite{gea} ist ein Verweis auf das Buch „A Theory
of Fun for Game Design“ in dem Raph Koster den Begriff „Spiel“ wie folgt definiert:

\begin{quote}
	„Ein Spiel ist eine interaktive Erfahrung, die dem Spieler ermöglicht, steigende Herausforderungen von Mustern, welche der Spieler im Laufe des Spieles lernt, zu meistern.“
	(Aus \cite{theoryoffun}, Zitat aus dem Englischen übersetzt)
\end{quote}
\subsection{Video-Spiele}

In der Videospielindustrie ist ein Spiel eine Software, die Bilder oder virtuelle dreidimensionale Objekte, die durch Spieler-Eingaben beeinflusst werden können, um ein vom Entwickler (oder genauer vom Spieldesigner) festgelegtes Ziel zu erreichen. 
Im Normalfall wird vom Spieler ein menschen- oder tierähnliches Wesen oder Fahrzeug durch ein Eingabegerät fortbewegt. Betont wird hier der Normalfall, da es verschiedene Genres oder Videospiel-Formen gibt. Hier ist die Kreativität des Entwicklers gefragt.

Gregory definiert in dem Buch „Game Engine Architecture“ ein Video-Spiel aus technischer Sicht als „soft real-time interactive agent-based computer simulations“ \cite{gea} (interaktive agentbasierte computergestützte Echtzeitsimulationen).
Doch wie kommt man zu diesem Begriff?

Die meisten Spiele bilden eine Simulation einer realen oder fiktiven Welt ab, die mathematisch modelliert wird. In einer Agent basierten Simulation interagieren verschiedene „Entities“ miteinander, weshalb auch objektorientierte Programmiersprachen (wie z.B. C++ oder Java) verwendet werden.
Solche interaktiven Videospiele sind zeitlich abhängige Simulationen, da sich Eigenschaften, wie zum Beispiel das Aussehen, in dieser fiktiven Welt mit der Zeit verändern. Außerdem sollte das Programm in Echtzeit auf die Eingaben des Spielers reagieren.

\subsection{Game-Engine}	
\label{engine}
Wenn man Spiele entwickeln möchte, so taucht früher oder später das Wort „Game-Engine“ auf. Einige Entwicklungsfirmen verweisen auch auf ihre Engine, um technisch besser da zu stehen und ihrem Spiel ein besseres Ansehen zu verschaffen. Heißt das aber, mit einer besseren Game-Engine kann man ein besseres Spiel entwickeln? Um dies heraus zu finden, sollte man sich erst einmal anschauen, wie solche Engines funktionieren und aufgebaut sind.

Frühere Spiele wie „Tetris“, „PacMan“ oder „Space-Invaders“ wurden immer „from scratch“  entwickelt. Die Entwickler besaßen also so gut wie gar kein Gerüst, bevor sie dieses Spiel programmierten. Da die meisten Spiele aber einen ähnlichen Aufbau besitzen, fiel auf, dass man immer einen ähnlichen Programm-Code verwenden könnte, um die Grundstruktur eines Spieles zu entwickeln. Vereinfacht verfügt ein Spiel über die Zentralbereiche der Grafik und die der Logik. Die Soundausgabe ist bei Spielen optional, da auch Spiele ohne Musik und Sound-Effekte existieren. Dennoch bieten die meisten Engines auch hierfür ein Grundgerüst.

Eine Game-Engine bietet dem Entwickler also diese Rahmenfunktionen, die seiner Software das Grundgerüst liefert. Dies heißt in der Softwareentwicklung „Framework“. 
Moderne Game-Engines sind heutzutage auf ein bestimmtes Genre oder auf bestimmte Aufgabenbereiche spezialisiert. So gibt es Game-Engines sowohl für 2D- als auch 3D-Spiele. Der „GameMaker“ von „YoyoGames“ ist zum Beispiel auf 2D-Spiele spezialisiert. Doch kann man auch Spiele in 3D mit ihm erstellen. Die „CryEngine“ hingegen ist hauptsächlich für 3D-Spiele konzipiert. (Für weitere Details zu diesem Thema siehe \ref{engines}.)


\section{Geschichte}
\subsection{Vergangenheit bis heute}
\label{vergangenheit}
Die ersten Konsolen auf dem Markt besaßen geringen Speicher. So mussten die Entwickler die Spiele immer speicherkalkuliert entwickeln, sodass sie möglichst klein und speichersparend waren. Da man also für jedes Spiel einen optimierten Code benötigte, waren Game-Engines von Drittanbietern nicht nötig.

Erst mit den 3D-Spielen und verbesserter Hardware wurden die Game-Engines von externen Unternehmen populär. Konsolen und PCs verfügten nun über genug Speicher und man musste nicht mehr auf jedes Byte achten. Die "`Angst"', man könne Speicher verschwenden oder ein zu großes Programm entwickeln, wurde mit der Zeit immer mehr zurück gedrängt.
Da die Entwickler nun auch Spiele herausbringen wollten, welche auf dem Markt besser ankommen und technisch fortschrittlicher sind, wurde der Code der Software um vieles komplexer.\todo{!}

So kam man schnell auf die Idee, ein System zu entwickeln, das für die Hauptbereiche eines Spiels ein Grundgerüst baut, die Software in Betrieb hält und das immer wieder verwendbar ist. Mit einer Game-Engine konnte man nun Zeit und Aufwand beim Programmieren sparen, mehr Zeit in die Optimierung des Spieles stecken und sich mehr Zeit für die Spielinhalte nehmen. 

\input{01einfuhrung/Zukunft}

\section{Engines}
\label{engines}
\subsection{GameMaker - YoyoGames}

Die zuerst für 2D-Animationen konzipierte Game-Engine "'GameMaker"` (In den Anfängen \textit{Animo}) wurde von Overmars erstmals im Jahre 1999 veröffentlicht und ist nun als "'GameMaker:Studio"` bekannt. Zunächst hat man sich bei der Entwicklung sehr stark auf die 2D-Spiele Entwicklung konzentriert. Mittlerweile kann man mit ihr sogar 3D-Spiele entwickeln. 

Das Programm ist momentan in drei verschiedenen Versionen erhältlich (Stand 2017). Darunter auch eine kostenlose Version mit beschränkten Funktionen.
Die Benutzeroberfläche ist sehr simpel gehalten und ist so konzipiert, dass sogar ein Leihe den Programmablauf eines Spieles durch einfaches "Drag'n'Drop" von Schaltflächen schnell erstellen und strukturieren kann. Zudem besitzt der "'GameMaker"` eine eigene Skript-Sprache, welche einige der höheren Programmiersprachen als Vorbild nimmt.

Die mit der GameMaker-Engine entwickelten Spiele werden mit verschiedene \textit{Resourcen} entwickelt;
Sprites, Sounds, Backgrounds, Paths, Scripts, Fonts, Timelines, Objects und Rooms. 
Auch wir wollten dies so ähnlich in unserem Designer handhaben und haben uns von dem Prinzip der Resourcen inspirieren lassen. \ref{designer}

Die Spiele können für diverse gängigen Betriebssysteme (Windows, Android, iOS ua.) aber auch für populäre Spielekonsolen (PlayStation, XBox ua.) kompiliert werden. 

\subsection{Unity3D - Unity Technologies}

"`Unity3D"' ist eine Game-Engine der Firma \textit{Unity Technologies}, welche erstmals im Jahre 2005 veröffentlicht wurde. Sie ist für die 3D-Spiele Entwicklung konzipiert, besitzt eine eigene 2D-Engine, ist weitaus komplexer als der "`GameMaker"' und bietet somit auch mehr Funktionen. 
Mit ihr kann man Spiele für Computer, aber auch Spielkonsolen, mobile Geräte und Webbrowser entwickeln. Sie ist für Windows, Linux (nur Beta) und OSX in vier verschiedenen Versionen erhältlich, unter denen sich auch eine kostenlose Version befindet. Das Programm beinhaltet zudem noch einen 3D-Terrain-Modellierer, Baum- und Pflanzen-Modell-Editor, Werkzeuge für Partikeleffekte und ein Tool für Bewegungssteuerung für Charaktere.

Die grafische Entwicklungsumgebung ähnelt sehr modernen 3D-Animationsprogrammen. Sie stellt im Hauptfenster eine Spiel-Szene dar und verfügt zudem über eine sogenannte \textit{Timeline}, in der der Zeitablauf des Spieles dargestellt wird. In Menüs kann man Parameter des Spieles verändern und sie für die Strukturierung des Spieles verwenden. 

Das Spiel-Scripting basiert auf dem \textit{Mono}-Compiler und bietet verschiedene Scriptsprachen, darunter auch C\# und eine vom Entwicklerteam eigens entwickelte Sprache "`UnityScript"'. Sie ist hauptsächlich für \textit{Cross-Platform} Programme gedacht. Das heißt, dass alle Scripte sofort für verschiedene Geräte exportiert werden können, ohne davor den Code an das Gerät anpassen zu müssen.

Eine aktuelle Szene besteht aus \textit{Game-Objects}, denen man verschiedene Eigenschaften (Materialien, Klänge, physikalische Eigenschaften, Skripte) zuordnen kann. Dies haben wir auch bei unserem Designer mittels Komponenten realisiert (siehe \cref{designer} oder \cref{entitysystem}).

Aufgrund dieser Komplexität ist die "`Unity3D-Engine"' eher für Fortgeschrittene geeignet und braucht weitaus mehr Einarbeitungszeit als der "`GameMaker"', sie bietet jedoch auch mehr Möglichkeiten.
Bei der Auswahl einer geeigneten Engine muss man abwägen, ob eine einfach zu bedienende Engine auch allen erforderlichen Entwicklungsmöglichkeiten bietet.

