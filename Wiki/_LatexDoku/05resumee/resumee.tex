\chapter{Resumé}

\section{Verbesserungsfähiges}
 Wie jedes Entwicklerteam hatten auch wir einige, nicht wenige Schwierigkeiten bei der Entwicklung unserer Software. Die wichtigsten Fehler und verbesserbaren Punkte haben wir hier dokumentiert.

\subsection{Software}
Der ursprüngliche Grund, warum wir den Designer entwickeln wollten, war das räumliche Darstellen von dreidimensional-renderbaren Entities und verschiedener Szenen vor dem Export in ein C++ Projekt. Handhaben wollten wir es ähnlich wie bei WPF: Eine Designer-Ansicht sollte die Übersicht und Optionen auf Entities liefern, welche mit einem Szenen-Editor editierbar wäre. Diese Ansicht sollte dem Nutzer die Ausgangslage des Spieles anzeigen.
Der Szenen-Editor sollte dem Nutzer ermöglichen, verschiedene Entity-Presets (siehe: \cref{entitysystem}) in eine Szene zu setzen. Die Szene würde dann mit XML- und Designer-spezifischen Skripten beschrieben und später wie Entities via Pipe (siehe\cref{pipe}) per Extension in Code umgewandelt werden.
Der Nutzer der Engine könnte dadurch selbständig ein komplett funktionsfähiges Programm generieren, ohne selbst programmieren zu müssen.
Wenn man nun einen solchen Skript-gesteuerten Szenen-Editor implementieren will, so müsste man auch einen komplexeren Text-Editor mit mehreren Funktionen implementieren. Er sollte an die Befehle für Skripte und XML angepasst werden und über automatische Wortvervollständigung für Befehle verfügen. Der Code müsste direkt bei der Eingabe auf Richtigkeit geprüft und Fehler sollten dem Nutzer angezeigt werden.
Um dem Nutzer ein besseres Spiel-Entwicklungserlebnis zu garantieren, könnte man den Entity-Editor in andockbare Fenster umlagern. So wäre es dem Nutzer möglich verschiedene Entities parallel zu bearbeiten.
Vieles der geplanten Ziele mussten aus Zeitmangel heraus gekürzt werden. (Grund: Siehe \cref{kleinanfangen})

\subsection{Team und Workflow}
\subsubsection{Planung}
Uns wurde schon relativ früh bewusst, dass die Planung bei einem größeren Projekt das A und O ist. Die Arbeitsschritte müssen am Anfang klar definiert sein, sodass man die Arbeit zum einen aufteilen und zum anderen planmäßig in einer festen Zeit fertigstellen kann. Bei unserem Projekt war die Planung gegen Anfang recht ausgefeilt, doch zählt nicht nur die Planung am Anfang des Projektes, sondern auch die Planung und Strukturierung während das Projekt am Laufen ist und die Kommunikation \ref{kommunikation} währenddessen. Es kam des öfteren vor, dass während der Entwicklung des Programmes bessere und effizientere Lösungsvorschläge in Frage kamen, als ursprünglich geplant war. Dies hatte oft auch eine Umstrukturierung des Projektes zur Folge. Als Beispiel nehme man das Entity-System(Siehe: \ref{entitysystem}). Zu Anfang wollten wir die ganzen Objekte aus Klassen erstellen, doch erwies sich dies als ungeeignet, da sonst verschiedene Objekte nur schwer miteinander interagieren könnten und außerdem müsste man jede Klasse einzeln definieren, was sehr viel Zeit beansprucht. 
Als wir das Entity-System nun eingeführt hatten, mussten wir natürlich das komplette Projekt dem Entity-System anpassen. Dort hatten wir einen Planungsdefizit.
                                                                                      
\subsubsection{Kommunikation}
\label{kommunikation}
Kommunikation ist das \textbf{wichtigste} bei einem komplexeren Projekt, bei dem mehrere Personen involviert sind. Um einen ununterbrochenen Arbeitsfluss zu garantieren, muss auch eine flüssige, detaillierte und verständliche Kommunikation vorhanden sein um Missverständnisse zu verhindern. Man muss immer bedenken, dass Mitarbeiter nicht in den Kopf anderer schauen können. So muss man seine Mitarbeiter immer über den eigenen Arbeitsfortschritt am laufenden halten.

\subsubsection{Klein anfangen}
\label{kleinanfangen}
Eines unserer größten Probleme war die Unterschätzung der von uns benötigten Zeit und die Überschätzung von uns selbst. So kam es, dass wir uns zu viele Funktionen überlegt hatten, die wir in unsere Lernleistung verarbeiten wollten, welche wir später aber des Zeitdruckes wegen wieder herausstreichen mussten. 
Wir haben uns des öfteren an mehreren Teilen des Programmes gleichzeitig aufgehalten. Dies hatte zwar zum einen den Vorteil, dass man immer am Arbeiten war und man so durch kleinere Erfolgserlebnisse die Motivation am Arbeiten nicht verlieren konnte. Dennoch hatte es den großen Nachteil, dass so wichtige Teile und Funktionen eines Programmes immer weiter bis zum Ende aufgeschoben wurden. Anstatt sich sofort diesen wichtigen Grundfunktionen zu widmen, steckten wir die Zeit so in Nebenfunktionen, die nicht essentiell wichtig für das Programm waren.
So wurde uns im Laufe des Projektes klar, dass es besser ist, mit einem kleiner konzipierten Programm anzufangen, bei dem die Grundfunktionen alle vollständig funktionsfähig sind und erst später, wenn das Grundgerüst und die Grundfunktionen eines Programmes stehen, man es ausbauen und verbessern kann. 

\section{Haben wir unser Ziel erreicht?}
Auch wenn viele der Funktionen, die wir uns zusätzlich vorgenommen hatten, der Zeit wegen aus dem Programm gefallen sind, haben wir dennoch unser Ziel erreicht. Wir haben uns vorgenommen eine Engine zu entwickeln, die dem Nutzer das Programmieren von Spielen erleichtert. Zudem wollten wir Tools entwickeln, die das Arbeiten mit der Engine vereinfachen und die Funktionen zur Unterstützung bieten.
Genau dies haben wir auch erreicht:
Wir haben eine funktionsfähige Engine "`\textit{from Scratch}"' geschaffen, mit der man ein komplettes Spiel, aber auch andere 3D-GUI-lastige Anwendungen programmieren kann. Zudem haben wir ein sehr nützliches Tool für die Programm-Entwicklung entwickelt, welches dem Entwickler eine Menge Code-Schreibarbeit abnimmt. 

\section{Was haben wir gelernt?}
Rückblickend können wir sagen, dass wir eine Menge durch diese besondere Lernleistung gelernt und auch verinnerlicht haben. Dies reicht von theoretischem bis hin zu praktischem Wissen. 
Die analytische Auseinandersetzung verschiedener Game-Engines (siehe \cref{engines}) und das Einlesen in Fachbücher zu diesem Thema hat uns die Tür in die Spiele- und Engine-Entwicklung geöffnet. Um nun 3D-Grafiken darzustellen mussten wir uns mit dem OpenGL Syntax(siehe \cref{opengl}) sowie Projektions- und Licht- Berechnungen (siehe \cref{rendering}) auseinander setzen.
Die Entwicklung des FM3D-Designers hat uns tiefe Einblicke in den C-Sharp Syntax ermöglicht. So konnten wir uns eine weitere Programmiersprache aneignen. Für das Projekte- und Dateien- Speichern, so wie es in unserem Designer geschieht, mussten wir uns den Umgang mit XML-Dateien (siehe \cref{extensiblemarkuplanguage}) und den dazugehörigen Bibliotheken aneignen.  Auch wurde unser Umgang mit dem C++ Syntax gefestigt. Zudem haben wir gelernt, wie man eine Extension für VisualStudio entwickelt und mit ihr kommunizieren kann.
